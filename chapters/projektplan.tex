\section{Projektplan}
\subsection{Teoretisk bakgrund}
Skriv en teoretisk bakgrund som sammanfattar forskningsläget och beskriver det man redan vet och inte vet om ämnesområdet. Börja ganska allmänt med ett generellt problem som sätter in er undersökning i ett större sammanhang. Bli sedan mer specifika och inriktade mot er frågeställning. Källhänvisa till böcker, artiklar, hemsidor etc. i texten. Skriv den teoretiska bakgrunden utan underrubriker.

Längd: ½ -1 A4 (Times New Roman, 12 p, 1,5 radavstånd)

\subsection{Syfte}
Beskriv här syftet, alltså varför ni vill göra just i den här undersökningen, vad ni vill uppnå med resultaten eller hur resultaten kan bidra till förståelsen av det generella problem ni skrivit om i den teoretiska bakgrunden. Här skriver ni vad ni vill uppnå med ert gymnasiearbete. Ni kan till exempel börja med: Syftet med vårt gymnasiearbete är att beskriva/förklara/utvärdera/förstå … Beskriv också här kort hur ni planerar att uppnå syftet med arbetet. Ni kan till exempel skriva: I det här arbetet kommer vi att undersöka …
Längd: Ett par meningar

\subsection{Frågeställning och hypotes}
Frågeställningen ska precisera syftet. Formulera frågeställningen som en fråga som ni kan svara på när ni har genomfört undersökningen. Ni kan till exempel börja med: I det här arbetet frågar vi om …
Det går inte alltid att ställa hypoteser, men om det går gör på följande sätt: Skriv er hypotes som ett möjligt svar på frågeställningen. Förklara därefter hur ni kom fram till hypotesen. Använd det ni vet om ämnesområdet som stöd för er förklaring. Källhänvisa. Börja med: Vår hypotes är … eller Vi har hypotesen att … Det är viktigt att ni skriver tydligt hur ni resonerade er fram till hypotesen med hjälp av den kunskap som redan finns om ämnesområdet i litteraturen. Ett sätt att göra det är att använda ord som visar orsakssamband: eftersom, sedan, genom, som resultat av, därför, följaktligen etc. Till exempel: Eftersom blodtrycket ökar efter ansträngning (källhänvisning), ställde vi hypotesen att … Om hypotesen inte går att testa direkt, formulera en testbar förutsägelse. Börja till exempel med: Om vår hypotes stämmer, borde undersökningen visa att …
Längd: Ett par stycken
\vspace{1em}
\hrule
\vspace{1em}
\textbf{Fråga:} ''Vad är den maximala hastigheten ett modelltåg i skala 1:40 kan uppnå vid en given kurvradie utan att spåra ur?''

\textbf{Hypotes:} Studien använder sig av en förenklad, ideell, modell av hur kraftsituationen ser ut på ett tåg i en sväng. Detta visas is \cref{fig:tag_krafter_sväng}. Den baseras på den fiktiva centrifugalkraften $F_c$ som används eftersom beräkningarna sker ur tågets accelererande referensram. Även tyngdkraften $F_g$ används. Följande samband gäller:
\begin{align}
    F_c &= ma_c = m\frac{v^2}{r} \\
    F_g &= mg \\
    M_g &= F_gl_g \\
    M_c &=  F_cl_c \\
    M_c &= M_g \implies \text{urspårning}
\end{align}

\begin{figure}[h!]
    \centering
    \includegraphics[width=0.33\textwidth]{fig/tåg_krafter.png}
    \caption{Kraftsituationen på tåget vid en vänstersväng.}
    \label{fig:tag_krafter_sväng}
\end{figure}

\subsection{Empirisk metod}
Skriv en kortfattad beskrivning av hur ni planerar att genomföra undersökningen och vilket material ni tänker att använda. Skriv också om, och i så fall hur, ni tänker använda kontroller och hur många upprepningar, replikat, ni kommer att ha. Källhänvisa om ni har hittat metoden i någon annans arbete. För att göra metodbeskrivningen tydlig, använd ord som: sedan, efter, därefter, slutligen, för det första, för det andra, för det tredje, etc. Skriv i löpande text och inte i en punktlista. Skriv om olika metoder under var sin underrubrik.
Längd: ½-1 A4
Analytisk metod
Beskriv hur ni planerar att analysera de data ni samlat in samt vilka beräkningar och formler ni då kommer att använda. Källhänvisa om ni har hittat analysmetoden i någon annans arbete.
Längd: Ett par stycken.

Materialen som kommer användas under undersökningen kommer vara: ett 3D printad tåg som väger ca 1,4 kg, en kamera som filmar i minst 240 FPS, ett stativ för att hålla kameran, rutat papper (med sidlängd 1 cm), 3D printade rälssegment. 

Undersökningen kommer ske genom att det först väljs en viss radie. Därefter kommer tåget placeras på toppen av en backe med bestämd höjd. Målet med detta är att ge tåget en bestämd hastighet baserad på potentiel energi överförd till kenetisk energi. Sedan släpps tåget och resultaten från kameran och om den spårade ur eller ej skrivs ned. Om tåget inte har spårat ur kommer tåget placeras på en högre höjd och denna process repeteras tills tåget spårar ur. När tåget spårat ur välj en ny radie på kurvan och hela processen börjar om från början. När alla radier har testats kommer datan skrivas in i en graf där hastigheten och radien är grafens axlar.



\subsection{Tidsplan}
Här anger ni vad som ska göras, när det ska göras och vem som gör det.