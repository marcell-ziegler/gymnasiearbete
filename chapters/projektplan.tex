\section{Projektplan}
\subsection{Teoretisk bakgrund}
Under ideell kurvtagning verkar två krafter på ett järnvägsfordon: centrifugalkraften $F_c$ och tyngdkraften $F_g$ \parencite{carlos}. I en icke-doserad kurva ser krafterna ut som i \cref{fig:tågkrafter}. Detta innebär även att
\begin{align}
    F_c &= \frac{v^2}{r} \\
    F_g &= mg
\end{align}
där $v$ är tågets hastighet tangent med kurvan, $r$ är kurvans radie, $m$ är tågets massa och tyngdaccelerationen $g\approx \SI{9.82}{\m\per\s\squared}$.

\begin{figure}[h]
    \centering
    \includegraphics[width=0.5\textwidth]{fig/placeholder.png}
    \caption{PLACEHOLDER. Tagen från \textcite{carlos}.}
    \label{fig:tågkrafter}
\end{figure}


\subsubsection{Kurvdosering och rälsförhöjning}
När en kurva doseras förändras kraftsituationen, som visat i \cref{fig:tågkrafter_doserad}. Doseringen kvantifieras som rälsförhöjning vilket uttrycker differensen i höjd mellan yttre räl och inre räl. Anordnad rälsförhöjning $h_a$ ges som
\begin{equation}
    h_a = h_\mathrm{yttre} - h_\mathrm{inre}.
\end{equation}
Med hjälp av detta kan lateralacceleration parallellt spårplanet beräknas enligt \parencite{carlos}
\begin{equation}
    a_y = \frac{v^2}{r}-g\cdot\sin{\varphi} = \frac{v^2}{r} - g\cdot \frac{h_a}{b}
    \label{eq:ay}
\end{equation}
där $b$ är spårvidden och $\varphi$ är vinkeln av spårplanet mot markplanet.

När kurvor konstrueras eftersträvar man lateral jämvikt. Om kraftresultanten $\vec F_\mathrm{res} = \vec F_c + \vec F_g$ är vinkelrät med spårplanet medför detta att den laterala acceleration $a_y=0$ vilket i sin tur medför lateral jämvikt \parencite{carlos}. När $a_y\neq0$ finns det risk för urspårning eftersom krafterna vid hjulet inte nödvändigtvis kan motstå den mot spårplanet horisontella komposanten av $\vec F_\mathrm{res}$.

När $a_y\neq 0$ råder det antingen rälsförhöjningsbrist eller rälsförhöjningsöverskott. Rälsförhöjningsbrist innebär den höjden som måste adderas till rälsförhöjningen för att uppnå jämvikt om fordonet har en acceleration utåt svängen. Rälsförhöjningsöverskott uppstår när fordonet upplever en acceleration inåt svängen eftersom $F_c$ är för liten. Rälsförhöjningsöverskottet är irrelevant för denna undersökning eftersom den undersöker höghastighetsurspårning där rälsförhöjningsbrist alltid uppstår.

För att beräkna rälsförhöjningsbristen $h_b$ krävs den anordnade rälsförhöjningen $h_a$ och jämviktsrälsförhöjningen $h_\mathrm{eq}$ som är \parencite{carlos}
\begin{equation}
    h_\mathrm{eq} =\frac{b}{g}\cdot \frac{v^2}{r}.
\end{equation}
Dessa används sedan för att ge rälsförhöjningsbrist som
\begin{equation}
    h_b=h_\mathrm{eq} - h_a
\end{equation}
där $h_\mathrm{eq} > h_a$ vilket är ett krav för rälsförhöjningsbrist.

\begin{figure}[h]
    \centering
    \includegraphics[width=0.5\textwidth]{fig/placeholder.png}
    \caption{PLACEHOLDER. Tagen från \textcite{carlos}.}
    \label{fig:tågkrafter_doserad}
\end{figure}

\subsubsection{Statliga säkerhetsdefinitioner}
Studien utgår från svenska järnvägar. Loket som modelleras är \textit{Siemens Vectron} vilket har en maximal tillåten rälsförhöjningsbrist på \SI{150}{\milli\m} \parencite{}. Enligt trafikverkets specifikation på svensk spårgeometri \parencite{} får detta fordon då uppleva $a_y\leq \SI{0.98}{\m\per\s\squared}$ i lateralacceleration.

Eftersom studien endast behandlar icke-doserade kurvor kommer $a_y=a_c$ alltid gälla där $a_c$ är centrifugalaccelerationen på tåget. Detta är eftersom $\varphi=\SI{0}{\degree}$ i en icke-doserad kurva (se \cref{eq:ay}). Med $a_y= \frac{v^2}{r}$ som givet av en centralrörelse kommer följande gälla:
\begin{equation}
    v=\sqrt{a_y r}
\end{equation}
samt även
\begin{equation}
    r=\frac{v^2}{a_y}.
    \label{eq:radie_lateral_acc}
\end{equation}

Svenska tåg färdas i regel med ca. $\SI{200}{\kilo\meter\per\hour} \approx \SI{55.56}{\m\per\s}$ på regionala sträckor. Sätts detta och $a_y=\SI{0.98}{\m\per\s\squared}$, för gränsfallet av fordonet, in i \cref{eq:radie_lateral_acc} ges $r\approx \SI{3800}{\m}$ vilket i studiens skala (1:40) blir $r\approx 3800/40\,\unit{\m} = \SI{80}{\m}$. Detta utgör grunden för valet av kurvradier i \cref{sec:metod}.

\begin{center}
    \hrule
    {\large\textbf{Notering:}} \\
    Mer kommer läggas till i detta kapitel framöver. Detta innefattar bl.a. mer om dynamiken av kurvtagning på en djupare nivå och möjligtvis mer matte vid behöv, men detta kräver läsning av en lärobok som vi endast idag fick tag på. Slutgiltig inlämning kommer att vara mer fyllig.
    \hrule
\end{center}

\subsection{Syfte}
Syftet med denna undersökning är att bekräfta tidigare idéer om säkra hastigheter i bestämda radier samt förklara sambandet mellan radier, hastighet och när ett tåg spårar ut. Detta kan sedan användas för att enklare planera och konstruera säkra järnvägar. Bättre förståelse av säkra kurvtagningshastigheter ger oss möjlighet att optimera de nya men även de befintliga järnvägarna för att minska restider. Det är ett stort problem att Sveriges järnvägar har många kurvor \parencite{} och denna förståelse kan bidra till planeringen av järnvägens rätande.

\subsection{Frågeställning}
''Vad är den maximala hastigheten ett modelltåg i skala 1:40 kan uppnå vid en given kurvradie utan att spåra ur?''

\subsection{Empirisk metod}
\label{sec:metod}
Ett 3d printad modelltåg i skala 1:40 (Modellerat efter en SIEMENS Vectron, men hjuprofil S1002 \parencite{}) med en vikt på \num{1.4} kg, en kamera som filmar i minst 240 FPS, ett stativ för att hålla kameran, rutat papper (vars rutor har sidlängd 1 cm) och 3D printade rälssegment i skala 1:40 (utefter UIC60 standard \parencite{}) med radier: 80, 40, 25, 12, \num{6.5}, \num{3.5}, 2, 1 (m).

Undersökningen kommer ske genom att det först väljs en specifik cirkelradie. Därefter kommer tåget placeras en bestämd höjd upp på en kulle. Anledningen till detta är att formeln för tågets potentiella energi då kan användas för att preliminärt beräkna hastigheten (sätt in formler). Om tåget inte spårar ur vid denna hastighet kommer det placeras högre upp i kullen tills gränsen hittas. Då kommer inspelningen från kameran med hjälp av det rutade pappret användas för att mer exakt bestämma hastigheten, denna hastighet tillsammans med radien kommer sedan sparas och användas i resultaten. Detta repeteras sedan med de andra bestämda radierna. När maxhastigheter har hittats för alla radier kommer fördelningen av vikten i tåget förflyttas genom att vikter i tåget placeras högre eller lägre ned.

När all data har samlats in kommer den användas för att skapa en regression med hastighet och radie som grafaxlar. Detta kommer göras separat för varje viktfördelning.

\subsection{Tidsplan}

\begin{center}
    \begin{tabular}{|c||l|}
        \hline
        v. 40 & Färdigt utkast för projektplanen \\\hline
        v. 41 & Deadline: utkast. Arbeta vidare med CAD \\\hline
        v. 42 & Bearbetning av CAD \\\hline
        v. 43 & Deadline: projektplan \\\hline
        v. 44 & Höstlov \\\hline
        v. 45 & Konstruktion av modellens första prototyp \\\hline
        v. 46 & Möjlighet att revidera modell \\\hline
        v. 47 & Simulering på KTH? Annars vidare revidering och/eller labb. \\\hline
        v. 48 & Labba \\\hline
        v. 49 & Dataanalys \\\hline
        v. 50 & Statistikworkshop / Dataanalys \\\hline
        v. 51--1 & Jullov \\\hline
        v. 2 & Genomgång: felanalys \\\hline
        v. 3 & Möjligtvis vidare analys annars börja skriva slutrapport \\\hline
        v. 4 & Skriv \\\hline
        v. 5 & Skriv \\\hline
        v. 6 & Skriv \\\hline
        v. 7 & Skriv \\\hline
        v. 8 & Skriv \\\hline
        v. 9 & Lov och/eller paniksrivning \\\hline
        v. 10 & Genomgång: opposition. (Panikskrivning?) \\\hline
        v. 11 & Deadline: rapport for opposition \\\hline
        v. 12 & Förberedelse av opposition  \\\hline
        v. 13 & Opposition \\\hline
        v. 14 & Opposition \\\hline
        v. 15 & Lov \\\hline
        v. 16 & Revidering av rapport \\\hline
        v. 17 & Gymnasiearbetets dag \\\hline
        v. 18 & Deadline: \textit{allt} \\\hline
    \end{tabular}
\end{center}