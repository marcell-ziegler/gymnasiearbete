\section{Projektplan}
\subsection{Teoretisk bakgrund}
Under ideell kurvtagning verkar två krafter på ett järnvägsfordon: centrifugalkraften $F_c$ och tyngdkraften $F_g$ \parencite{carlos}. I en icke-doserad kurva ser krafterna ut som i \cref{fig:tågkrafter}. Detta innebär även att
\begin{align}
    F_c &= \frac{v^2}{r} \\
    F_g &= mg
\end{align}
där $v$ är tågets hastighet tangent med kurvan, $r$ är kurvans radie, $m$ är tågets massa och tyngdaccelerationen $g\approx \SI{9.82}{\m\per\s\squared}$.

\begin{figure}[h]
    \centering
    \includegraphics[width=0.5\textwidth]{fig/placeholder.png}
    \caption{PLACEHOLDER. Tagen från \textcite{carlos}.}
    \label{fig:tågkrafter}
\end{figure}


\subsubsection{Kurvdosering och rälsförhöjning}
När en kurva doseras förändras kraftsituationen, som visat i \cref{fig:tågkrafter_doserad}. Doseringen kvantifieras som rälsförhöjning vilket uttrycker differensen i höjd mellan yttre räl och inre räl. Anordnad rälsförhöjning $h_a$ ges som
\begin{equation}
    h_a = h_\mathrm{yttre} - h_\mathrm{inre}.
\end{equation}
Med hjälp av detta kan lateralacceleration parallellt spårplanet beräknas enligt \parencite{carlos}
\begin{equation}
    a_y = \frac{v^2}{r}-g\cdot\sin{\varphi} = \frac{v^2}{r} - g\cdot \frac{h_a}{b}
    \label{eq:ay}
\end{equation}
där $b$ är spårvidden och $\varphi$ är vinkeln av spårplanet mot markplanet.

När kurvor konstrueras eftersträvar man lateral jämvikt. Om kraftresultanten $\vec F_\mathrm{res} = \vec F_c + \vec F_g$ är vinkelrät med spårplanet medför detta att den laterala acceleration $a_y=0$ vilket i sin tur medför lateral jämvikt \parencite{carlos}. När $a_y\neq0$ finns det risk för urspårning eftersom krafterna vid hjulet inte nödvändigtvis kan motstå den mot spårplanet horisontella komposanten av $\vec F_\mathrm{res}$.

När $a_y\neq 0$ råder det antingen rälsförhöjningsbrist eller rälsförhöjningsöverskott. Rälsförhöjningsbrist innebär den höjden som måste adderas till rälsförhöjningen för att uppnå jämvikt om fordonet har en acceleration utåt svängen. Rälsförhöjningsöverskott uppstår när fordonet upplever en acceleration inåt svängen eftersom $F_c$ är för liten. Rälsförhöjningsöverskottet är irrelevant för denna undersökning eftersom den undersöker höghastighetsurspårning där rälsförhöjningsbrist alltid uppstår.

För att beräkna rälsförhöjningsbristen $h_b$ krävs den anordnade rälsförhöjningen $h_a$ och jämviktsrälsförhöjningen $h_\mathrm{eq}$ som är \parencite{carlos}
\begin{equation}
    h_\mathrm{eq} =\frac{b}{g}\cdot \frac{v^2}{r}.
\end{equation}
Dessa används sedan för att ge rälsförhöjningsbrist som
\begin{equation}
    h_b=h_\mathrm{eq} - h_a
\end{equation}
där $h_\mathrm{eq} > h_a$ vilket är ett krav för rälsförhöjningsbrist.

\begin{figure}[h]
    \centering
    \includegraphics[width=0.5\textwidth]{fig/placeholder.png}
    \caption{PLACEHOLDER. Tagen från \textcite{carlos}.}
    \label{fig:tågkrafter_doserad}
\end{figure}

\subsubsection{Statliga säkerhetsdefinitioner}
Studien utgår från svenska järnvägar. Loket som modelleras är \textit{Siemens Vectron} vilket har en maximal tillåten rälsförhöjningsbrist på \SI{150}{\milli\m} \parencite{}. Enligt trafikverkets specifikation på svensk spårgeometri \parencite{} får detta fordon då uppleva $a_y\leq \SI{0.98}{\m\per\s\squared}$ i lateralacceleration.

Eftersom studien endast behandlar icke-doserade kurvor kommer $a_y=a_c$ alltid gälla där $a_c$ är centrifugalaccelerationen på tåget. Detta är eftersom $\varphi=\SI{0}{\degree}$ i en icke-doserad kurva (se \cref{eq:ay}). Med $a_y= \frac{v^2}{r}$ som givet av en centralrörelse kommer följande gälla:
\begin{equation}
    v=\sqrt{a_y r}
\end{equation}
samt även
\begin{equation}
    r=\frac{v^2}{a_y}.
    \label{eq:radie_lateral_acc}
\end{equation}

Svenska tåg färdas i regel med ca. $\SI{200}{\kilo\meter\per\hour} \approx \SI{55.56}{\m\per\s}$ på regionala sträckor. Sätts detta och $a_y=\SI{0.98}{\m\per\s\squared}$, för gränsfallet av fordonet, in i \cref{eq:radie_lateral_acc} ges $r\approx \SI{3800}{\m}$ vilket i studiens skala (1:40) blir $r\approx 3800/40\,\unit{\m} = \SI{80}{\m}$. Detta utgör grunden för valet av kurvradier i \cref{sec:metod}.

\begin{center}
    \hrule
    {\large\textbf{Notering:}} \\
    Mer kommer läggas till i detta kapitel framöver. Detta innefattar bl.a. mer om dynamiken av kurvtagning på en djupare nivå och möjligtvis mer matte vid behöv, men detta kräver läsning av en lärobok som vi endast idag fick tag på. Slutgiltig inlämning kommer att vara mer fyllig.
    \hrule
\end{center}

\subsection{Syfte}
Beskriv här syftet, alltså varför ni vill göra just i den här undersökningen, vad ni vill uppnå med resultaten eller hur resultaten kan bidra till förståelsen av det generella problem ni skrivit om i den teoretiska bakgrunden. Här skriver ni vad ni vill uppnå med ert gymnasiearbete. Ni kan till exempel börja med: Syftet med vårt gymnasiearbete är att beskriva/förklara/utvärdera/förstå … Beskriv också här kort hur ni planerar att uppnå syftet med arbetet. Ni kan till exempel skriva: I det här arbetet kommer vi att undersöka …
Längd: Ett par meningar

\subsection{Frågeställning}
''Vad är den maximala hastigheten ett modelltåg i skala 1:40 kan uppnå vid en given kurvradie utan att spåra ur?''

\subsection{Empirisk metod}
\label{sec:metod}
Materialen som kommer användas under undersökningen kommer vara: ett 3D printad tåg som väger ca 1,4 kg, en kamera som filmar i minst 240 FPS, ett stativ för att hålla kameran, rutat papper (med sidlängd 1 cm), 3D printade rälssegment.

Undersökningen kommer ske genom att det först väljs en viss radie. Därefter kommer tåget placeras på toppen av en backe med bestämd höjd. Målet med detta är att ge tåget en bestämd hastighet baserad på potentiell energi överförd till kinetisk energi. Sedan släpps tåget och resultaten från kameran och om den spårade ur eller ej skrivs ned. Om tåget inte har spårat ur kommer tåget placeras på en högre höjd och denna process repeteras tills tåget spårar ur. När tåget spårat ur välj en ny radie på kurvan och hela processen börjar om från början. Utöver detta kommer vi även variera masscentrumet genom att placera vikter på olika höjder i modelltåget, detta kommer ske i flera replikat. När alla radier har testats kommer data skrivas in i en graf där hastigheten och radien är grafens axlar.



\subsection{Tidsplan}

\begin{center}
    \begin{tabular}{|c||l|}
        \hline
        v. 40 & Färdigt utkast för projektplanen \\\hline
        v. 41 & Deadline: utkast. Arbeta vidare med CAD \\\hline
        v. 42 & Bearbetning av CAD \\\hline
        v. 43 & Deadline: projektplan \\\hline
        v. 44 & Höstlov \\\hline
        v. 45 & Konstruktion av modellens första prototyp \\\hline
        v. 46 & Möjlighet att revidera modell \\\hline
        v. 47 & Simulering på KTH? Annars vidare revidering och/eller labb. \\\hline
        v. 48 & Labba \\\hline
        v. 49 & Dataanalys \\\hline
        v. 50 & Statistikworkshop / Dataanalys \\\hline
        v. 51--1 & Jullov \\\hline
        v. 2 & Genomgång: felanalys \\\hline
        v. 3 & Möjligtvis vidare analys annars börja skriva slutrapport \\\hline
        v. 4 & Skriv \\\hline
        v. 5 & Skriv \\\hline
        v. 6 & Skriv \\\hline
        v. 7 & Skriv \\\hline
        v. 8 & Skriv \\\hline
        v. 9 & Lov och/eller paniksrivning \\\hline
        v. 10 & Genomgång: opposition. (Panikskrivning?) \\\hline
        v. 11 & Deadline: rapport for opposition \\\hline
        v. 12 & Förberedelse av opposition  \\\hline
        v. 13 & Opposition \\\hline
        v. 14 & Opposition \\\hline
        v. 15 & Lov \\\hline
        v. 16 & Revidering av rapport \\\hline
        v. 17 & Gymnasiearbetets dag \\\hline
        v. 18 & Deadline: \textit{allt} \\\hline
    \end{tabular}
\end{center}