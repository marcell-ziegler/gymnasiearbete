\section{Projektplan}
\subsection{Teoretisk bakgrund}
Under ideell kurvtagning verkar två krafter på ett järnvägsfordon: centrifugalkraften $F_c$ och tyngdkraften $F_g$ \parencite{carlos}. I en icke-doserad kurva ser krafterna ut som i \cref{fig:tågkrafter}. Detta innebär även att
\begin{align}
    F_c &= \frac{v^2}{r} \\
    F_g &= mg
\end{align}
där $v$ är tågets hastighet tangent med kurvan, $r$ är kurvans radie, $m$ är tågets massa och tyngdaccelerationen $g\approx \SI{9.82}{\m\per\s\squared}$.

\begin{figure}[h]
    \centering
    \includegraphics[width=0.5\textwidth]{fig/placeholder.png}
    \caption{PLACEHOLDER. Tagen från \textcite{carlos}.}
    \label{fig:tågkrafter}
\end{figure}


\subsubsection{Kurvdosering och rälsförhöjning}
När en kurva doseras förändras kraftsituationen, som visat i \cref{fig:tågkrafter_doserad}. Doseringen kvantifieras som rälsförhöjning vilket uttrycker differensen i höjd mellan yttre räl och inre räl. Anordnad rälsförhöjning $h_a$ ges som
\begin{equation}
    h_a = h_\mathrm{yttre} - h_\mathrm{inre}.
\end{equation}
Med hjälp av detta kan lateralacceleration parallellt spårplanet beräknas enligt \parencite{carlos}
\begin{equation}
    a_y = \frac{v^2}{r}-g\cdot\sin{\varphi} = \frac{v^2}{r} - g\cdot \frac{h_a}{b}
\end{equation}
där $b$ är spårvidden och $\varphi$ är vinkeln av spårplanet mot markplanet.

När kurvor konstrueras eftersträvar man lateral jämvikt. Om kraftresultanten $\vec F_\mathrm{res} = \vec F_c + \vec F_g$ är vinkelrät med spårplanet medför detta att den laterala acceleration $a_y=0$ vilket i sin tur medför lateral jämvikt \parencite{carlos}. När $a_y\neq0$ finns det risk för urspårning eftersom krafterna vid hjulet inte nödvändigtvis kan motstå den mot spårplanet horisontella komposanten av $\vec F_\mathrm{res}$.

När $a_y\neq 0$ råder det antingen rälsförhöjningsbrist eller rälsförhöjningsöverskott. Rälsförhöjningsbrist innebär den höjden som måste adderas till rälsförhöjningen för att uppnå jämvikt om fordonet har en acceleration utåt svängen. Rälsförhöjningsöverskott uppstår när fordonet upplever en acceleration inåt svängen eftersom $F_c$ är för liten. Rälsförhöjningsöverskottet är irrelevant för denna undersökning eftersom den undersöker höghastighetsurspårning där rälsförhöjningsbrist alltid uppstår.

För att beräkna rälsförhöjningsbrist krävs den anordnade rälsförhöjningen $h_a$ och den teoretiska rälsförhöjningen


\begin{figure}[h]
    \centering
    \includegraphics[width=0.5\textwidth]{fig/placeholder.png}
    \caption{PLACEHOLDER. Tagen från \textcite{carlos}.}
    \label{fig:tågkrafter_doserad}
\end{figure}

\subsection{Syfte}
Syftet med denna undersökning är att bekräfta tidigare idéer om säkra hastigheter i bestämda radier samt förklara sambandet mellan radier, hastighet och när ett tåg spårar ut. Detta kan sedan användas för att enklare planera och konstruera säkra järnvägar. Bättre förståelse av säkra kurvtagningshastigheter ger oss möjlighet att optimera de nya men även de befintliga järnvägarna för att minska restider. Det är ett stort problem att Sveriges järnvägar har många kurvor \parencite{} och denna förståelse kan bidra till planeringen av järnvägens rätande. 

\subsection{Frågeställning}
''Vad är den maximala hastigheten ett modelltåg i skala 1:40 kan uppnå vid en given kurvradie utan att spåra ur?''

\subsection{Empirisk metod}
Ett 3d printad modelltåg i skala 1:40 (Modellerat efter en SIEMENS Vectron, men hjuprofil S1002 \parencite{}) med en vikt på \num{1.4} kg, en kamera som filmar i minst 240 FPS, ett stativ för att hålla kameran, rutat papper (vars rutor har sidlängd 1 cm) och 3D printade rälssegment i skala 1:40 (utefter UIC60 standard \parencite{}) med radier: 80, 40, 25, 12, \num{6.5}, \num{3.5}, 2, 1 (m). 

Undersökningen kommer ske genom att det först väljs en specifik cirkelradie. Därefter kommer tåget placeras en bestämd höjd upp på en kulle. Anledningen till detta är att formeln för tågets potentiella energi då kan användas för att preliminärt beräkna hastigheten (sätt in formler). Om tåget inte spårar ur vid denna hastighet kommer det placeras högre upp i kullen tills gränsen hittas. Då kommer inspelningen från kameran med hjälp av det rutade pappret användas för att mer exakt bestämma hastigheten, denna hastighet tillsammans med radien kommer sedan sparas och användas i resultaten. Detta repeteras sedan med de andra bestämda radierna. När maxhastigheter har hittats för alla radier kommer fördelningen av vikten i tåget förflyttas genom att vikter i tåget placeras högre eller lägre ned. 

När all data har samlats in kommer den användas för att skapa en regression med hastighet och radie som grafaxlar. Detta kommer göras separat för varje viktfördelning. 

\subsection{Tidsplan}
vecka 40:
göra klart Projektplans utkastet och beställa kullager samt axlar till tågmodellen

Vecka 41: senaste dagen att lämna in projektplanen (måndag kl 1200) samt renskrivning av projektplan och mer inläsning av fysikaliska och matematiska modeller. utöver detta kada klart tågmodellen

vecka 42: samma som tidigare

vecka 43: inlämning av slutlig Projektplan

Vecka 44: höstlov

vecka 45: slutliga förberedelser för labben och ihopsättning/ kontrollering av modellens kvalite

vecka 46: om modellen visar sig vara av sämre kvalite; bygg om modellen, annars påbörja labben

vecka 47: simulering av experiment på KTH om tiden funkar för båda grupper, annars fortsättning med experiment

vecka 48: analys av data

vecka 49: formulera resultat

vecka 50: workshop statistik med excel

vecka 51-1: lov

vecka 2: genomgång felanalys

vecka 3: fortsätt bearbeta resultat och börja med att skriva ned dem

vecka 4: skriva resultat och diskussion

vecka 5: genomgång av naturvetenskaplig rapport och börja skriva resultat och diskussion

vecka 6: fortsätt skriva och börja med abstrakt när vi blir klara med tidigare delar

vecka 7: fortsätt skriva

vecka 8: fortsätt skriva

vecka 9: lov/kris skrivning

vecka 10: genomgång av hur oppositionen går till samt skrivning

vecka 11: inlämning av raport inför oppositionen

vecka 12: förberedelse av oppositionen

vecka 13: oppositioner

vecka 14: oppositioner

vecka 15: lov

vecka 16: revidering av rapport

vecka 17: gymnasiearbetes dag med åk 2

vecka 18: slutinlämning av rapport och allt annat (slut)